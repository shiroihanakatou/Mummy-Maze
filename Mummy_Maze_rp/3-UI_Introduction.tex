\chapter{UI INTRODUCTION}

\section{Sơ lược giao diện game}
\subsection{Tổng quan trò chơi}
Đồ án \textbf{Mummy Maze} là một ứng dụng trò chơi giải đố chiến thuật được phát triển dựa trên cảm hứng từ tựa game kinh điển của hãng PopCap. Trò chơi đặt người chơi vào bối cảnh các kim tự tháp Ai Cập huyền bí, nơi họ phải vận dụng khả năng tư duy logic để vượt qua các mê cung đầy rẫy nguy hiểm.

\subsection{Công nghệ sử dụng}
Đồ án được triển khai bằng ngôn ngữ lập trình \texttt{Python} kết hợp với thư viện \texttt{Pygame} và các tài nguyên đồ họa và Hoạt họa từ \href{https://www.spriters-resource.com/}{The Spriters Resource} và  \href{https://www.myabandonware.com/game/mummy-maze-deluxe-lm3}{My Abandonware}.
\begin{itemize}
    \item \textbf{Thư viện Pygame:} Là bộ công cụ mã nguồn mở hỗ trợ phát triển các ứng dụng đa phương tiện và trò chơi 2D. \texttt{Pygame} được lựa chọn nhờ khả năng xử lý đồ họa mượt mà, quản lý sự kiện bàn phím/chuột tối ưu và hệ thống âm thanh linh hoạt, giúp xây dựng cấu trúc trò chơi theo hướng đối tượng (OOP) một cách hiệu quả.
    
    \item \textbf{Tài nguyên đồ họa và Hoạt họa:} 
    Toàn bộ hệ thống hình ảnh được trích xuất và tham khảo từ hai nguồn chính \href{https://www.spriters-resource.com/}{The Spriters Resource} và  \href{https://www.myabandonware.com/game/mummy-maze-deluxe-lm3}{My Abandonware}:
    \begin{itemize}
        \item \textit{\href{https://www.myabandonware.com/game/mummy-maze-deluxe-lm3}{My Abandonware}:} cung cấp các graphics và sounds \textit{Mummy Maze Deluxe}.
        \item \textit{\href{https://www.spriters-resource.com/}{The Spriters Resource}:} tạo các nhân vật người chơi dùng trong game
    \end{itemize}

    \item \textbf{Logic điều khiển:} Hệ thống quản lý trạng thái để chuyển đổi giữa các màn hình như \textit{Login, Register, Menu, Game Mode, Playing...}
    
\end{itemize}



\section{Các chức năng của game:}
    \subsection{Đồ họa và hoạt ảnh}
Dưới đây là danh sách các tài nguyên đồ họa chính được trích xuất và sử dụng trong trò chơi, đảm bảo tái hiện phong cách đặc trưng của phiên bản gốc \href{https://www.spriters-resource.com/}{The Spriters Resource} và  \href{https://www.myabandonware.com/game/mummy-maze-deluxe-lm3}{My Abandonware}:

\begin{itemize}
    \item \textbf{Nhân vật:} Có thể chọn nhân vật mặc định (nhà thàm hiểm) hoặc nhân vật đã mua được trong shop (có tất cả 11 loại nhân vật).
    \begin{figure}[H]
        \centering
        \begin{minipage}{0.3\textwidth}
            \centering
            \includegraphics[width=0.8\linewidth]{images/explorer.png}
            \caption*{Nhà thám hiểm}
        \end{minipage}
        \begin{minipage}{0.3\textwidth}
            \centering
            \includegraphics[width=0.8\linewidth]{images/char_.png}
            \caption*{Nhân vật từ shop}
        \end{minipage}
        
    \end{figure}

    \item \textbf{Quái vật:} Tấn công người chơi (Có thể nhiều hơn 1 loại quái vật trong 1 màn chơi).
    \begin{figure}[H]
        \centering
        \begin{minipage}{0.3\textwidth}
            \centering
            \includegraphics[width=0.8\linewidth]{images/w_mm.png}
            \caption*{Xác ướp trắng}
        \end{minipage}
        \begin{minipage}{0.3\textwidth}
            \centering
            \includegraphics[width=0.8\linewidth]{images/r_mm.png}
            \caption*{Xác ướp đỏ}
        \end{minipage}
        \begin{minipage}{0.3\textwidth}
            \centering
            \includegraphics[width=0.8\linewidth]{images/scop.png}
            \caption*{Bọ cạp}
        \end{minipage}
        \caption{Các loại quái vật trong mê cung.}
    \end{figure}

    \item \textbf{Chướng ngại vật và Vật phẩm:} Tường và hàng rào cản trở việc di chuyển. Bẫy hình đầu lâu có thể giết người chơi. Chìa khóa có thể dùng để mở hoặc đóng hàng rào.
    \begin{figure}[H]
        \centering
        \begin{minipage}{0.24\textwidth}
            \centering
            \includegraphics[width=0.6\linewidth]{images/wall.png}
            \caption*{Tường đá}
        \end{minipage}
        \begin{minipage}{0.24\textwidth}
            \centering
            \includegraphics[width=0.6\linewidth]{images/gate.png}
            \caption*{Hàng rào}
        \end{minipage}
        \begin{minipage}{0.24\textwidth}
            \centering
            \includegraphics[width=0.6\linewidth]{images/key.png}
            \caption*{Chìa khóa}
        \end{minipage}
        \begin{minipage}{0.24\textwidth}
            \centering
            \includegraphics[width=0.6\linewidth]{images/trap.png}
            \caption*{Bẫy/Đầu lâu}
        \end{minipage}
        \caption{Các chướng ngại vật và vật phẩm.}
    \end{figure}
\end{itemize}

    \subsection{Hệ thống Quản lý tài khoản:}

        \subsubsection{Đăng ký tài khoản}
        Nếu chưa có tài khoản, người chơi có thể tạo tài khoản mới ở đây
        \begin{figure}[H]
            \centering
            \includegraphics[width=0.4\linewidth]{images/ui_register.png} 
            \caption{Màn hình Đăng ký tài khoản mới.}
            \label{fig:register_screen}
        \end{figure}

        \subsubsection{Đăng nhập}
        Khi đã có tài khoản, người chơi sẽ nhập thông tin tài khoản tại đây để truy cập vào trò chơi.
        \begin{figure}[H]
            \centering
            \includegraphics[width=0.4\linewidth]{images/ui_login.png} 
            \caption{Màn hình Đăng nhập.}
            \label{fig:login_screen}
        \end{figure}

        \subsubsection{Chức năng Chơi với tư cách khách (Play as Guest)}
        Đây là tính năng hỗ trợ người dùng trải nghiệm nhanh trò chơi mà không cần thông qua các bước khởi tạo tài khoản.

        \begin{itemize}
            \item \textbf{Mô tả:} Người chơi chỉ cần nhấn vào nút "Play as Guest" và chọn tài khoản guest có sẵn thì sẽ vào được game.
                \begin{itemize}
                \item Hệ thống sẽ cấp một định danh tạm thời cho phiên chơi đó.
                \item \textit{Hạn chế:} Kết quả chơi, điểm số hoặc tiến trình vượt ải sẽ không được lưu lại bền vững trong cơ sở dữ liệu sau khi xóa ứng dụng.
            \end{itemize}
        \end{itemize}

        \begin{figure}[H]
            \centering
            \includegraphics[width=0.4\linewidth]{images/ui_guest.png} 
            \caption{Chế độ khách.}
            \label{fig:guest_mode}
        \end{figure}


\subsection{Giao diện Menu chính}
    Sau khi truy cập thành công vào trò chơi, người chơi sẽ được đưa đến màn hình Menu chính. Đây là trung tâm điều hướng của toàn bộ ứng dụng.

    \begin{figure}[H]
        \centering
        \includegraphics[width=0.4\linewidth]{images/ui_main_menu.png} 
        \caption{Giao diện Menu chính.}
        \label{fig:main_menu}
    \end{figure}

    \begin{itemize}
        \item \textbf{Play :} Chuyển đến màn hình lựa chọn chế độ chơi (Game Mode).
            \begin{figure}[H]
                \centering
                \includegraphics[width=0.4\linewidth]{images/ui_game_mode.png} 
                \caption{Các chế độ game.}
                \label{fig:game_mode}
            \end{figure}

        \item \textbf{Tutorial :} Cung cấp các chỉ dẫn trực quan về cơ chế di chuyển theo lượt và các quy tắc đặc thù trong mê cung, giúp người chơi mới dễ dàng tiếp cận trò chơi.
            \begin{figure}[H]
                \centering
                \includegraphics[width=0.4\linewidth]{images/ui_tutorial.png} 
                \caption{Hướng dẫn.}
                \label{fig:tutorial}
            \end{figure}
        \item \textbf{Character :} Khu vực shop nhân vật. Tại đây, người chơi có thể chọn nhân vật là nhà thám hiểm mặc định hoặc các nhân vật đã mua bằng điểm.
            \begin{figure}[H]
                \centering
                \begin{minipage}{0.4\textwidth}
                    \centering
                    \includegraphics[width=\textwidth]{images/ui_character_shop_1.png} 
            
                \end{minipage}\hfill
                \begin{minipage}{0.4\textwidth}
                    \centering
                    \includegraphics[width=\textwidth]{images/ui_character_shop_2.png} 
                    
                \end{minipage}
                \caption{Character shop}
            \end{figure}
        \item \textbf{LeaderBoard :} Hiển thị danh sách các "nhà thám hiểm" xuất sắc nhất. Chức năng này thúc đẩy tính cạnh tranh giữa các người chơi thông qua việc lưu trữ và so sánh điểm số từ cơ sở dữ liệu.
            \begin{figure}[H]
                \centering
                \includegraphics[width=0.4\linewidth]{images/ui_leaderboard.png} 
                \caption{Bảng xếp hạng.}
                \label{fig:guest_mode}
            \end{figure}
    \end{itemize}
            
\subsection{Lựa chọn chế độ chơi} 

Sau khi người chơi nhấn chọn \textbf{Play} tại Menu chính, hệ thống sẽ chuyển sang giao diện lựa chọn chế độ chơi. Việc phân chia này giúp người chơi có những trải nghiệm khác nhau về mặt chiến thuật và không gian:

\begin{itemize}
    \item \textbf{Classic Mode:} 
    Trong chế độ này, người chơi sẽ đối mặt với các thử thách được phân cấp rõ rệt về độ khó. Hệ thống cung cấp 4 mức độ để người chơi lựa chọn:
    \begin{itemize}
        \item \textbf{Easy (Mê cung 6x6):} Dành cho người mới bắt đầu để làm quen với quy luật di chuyển.
        \item \textbf{Medium (Mê cung 8x8):} Tăng số lượng vật cản và vị trí xuất phát của kẻ thù khó khăn hơn.
        \item \textbf{Hard (Mê cung 10x10):} Kết hợp nhiều loại xác ướp và bẫy trong một diện tích hẹp.
        \item \textbf{Impossible (Không tưởng):} Thử thách cực hạn, yêu cầu người chơi phải điều hướng tối ưu để không bị dồn vào đường cùng.
    \end{itemize}
    \begin{figure}[H]
    \centering
    \includegraphics[width=0.4\linewidth]{images/ui_classic_difficulties.png} 
    \caption{Các mức độ thử thách trong chế độ Classic.}
    \label{fig:classic_levels}
    \end{figure}

    \item \textbf{Adventure Mode:} 
    \begin{itemize}
        \item Mở rộng trải nghiệm với các bản đồ có cấu trúc phức tạp và đa dạng về môi trường (Hầm mộ đá, Kim tự tháp vàng...).
        \item Bản đồ sẽ gồm các levels với mỗi level chơi sẽ gồm 15 màn. Mỗi màn chơi xong, sẽ đánh dấu vào 1 ô trong 15 ô của hình tam giác.
    \end{itemize}
    \begin{figure}[H]
    \centering
    \includegraphics[width=0.4\linewidth]{images/ui_adventure_worldmap.png} 
    \caption{Giao diện Bản đồ thế giới trong chế độ Adventure.}
    \label{fig:adventure_map}
    \end{figure}
\end{itemize}


\subsection{Giao diện Màn hình chơi}

Giao diện chơi game được thiết kế đồng nhất về mặt hình họa nhưng có sự phân hóa về chức năng nút bấm để phù hợp với từng mục tiêu chơi.

\subsubsection{Giao diện Chế độ Classic}
Trong chế độ Classic, dải menu bên trái cung cấp các công cụ tối ưu cho việc giải đố liên tục. Đặc biệt là nút \textbf{NEW GAME} giúp người chơi nhanh chóng chuyển sang các thử thách mới.

\begin{figure}[H]
    \centering
    \includegraphics[width=0.4\linewidth]{images/ui_classic_play.png} 
    \caption{Giao diện màn hình chơi chế độ Classic.}
    \label{fig:play_classic}
\end{figure}

\subsubsection{Giao diện Chế độ Adventure}
Đối với chế độ Adventure, hệ thống điều khiển được thay đổi để phục vụ tính chất khám phá. Nút \textbf{NEW GAME} được thay thế bằng biểu tượng hình kim tự tháp hiển thị tiến trình game.

\begin{figure}[H]
    \centering
    \includegraphics[width=0.4\linewidth]{images/ui_adventure_play.png} 
    \caption{Giao diện màn hình chơi chế độ Adventure.}
    \label{fig:play_adventure}
\end{figure}

\subsubsection{So sánh các nút chức năng chính}
\begin{itemize}
    \item \textbf{RESTART/UNDO:} Có mặt ở cả hai chế độ để hỗ trợ người chơi sửa sai trong các bước đi logic.
    \item \textbf{Phím di chuyển ảo:} Xuất hiện ngay dưới chân nhân vật (mũi tên xanh) giúp người chơi điều khiển trực quan bằng chuột hoặc cảm ứng.
    \item \textbf{Khu vực hiển thị:} Cả hai đều hiển thị rõ ràng vị trí nhân vật, xác ướp và cửa thoát hiểmnằm ở góc dưới bên phải lưới.
    \item \textbf{OPTIONS:} Dùng để hiển thị màn hình điều chỉnh độ lớn âm thanh nền, âm thanh hiệu ứng, tốc độ di chuyển nhân vật.
    \item \textbf{EXIT:} Khi bấm vào, người chơi sẽ thoát về màn hình menu khi bấm "YES" và lưu lại tiến trình của người chơi.
            \begin{figure}[H]
                \centering
                \begin{minipage}{0.45\textwidth}
                    \centering
                    \includegraphics[width=\textwidth]{images/ui_options.png} 
                    \caption{OPTIONS}
                \end{minipage}\hfill
                \begin{minipage}{0.45\textwidth}
                    \centering
                    \includegraphics[width=\textwidth]{images/ui_exit.png} 
                    \caption{EXIT}
                \end{minipage}
            
            \end{figure}
\end{itemize}
\subsection{Giao diện khi thất bại}
\subsubsection{Các trạng thái khi nhân vật bị chết}

\begin{figure}[H]
    \centering
    % Hàng 1: Chết do quái vật
    \begin{minipage}{0.3\linewidth}
        \centering
        \includegraphics[width=\linewidth]{images/death_white_mummy.png}
        \caption*{Do xác ướp trắng}
    \end{minipage} \hfill
    \begin{minipage}{0.3\linewidth}
        \centering
        \includegraphics[width=\linewidth]{images/death_red_mummy.png}
        \caption*{Do xác ướp đỏ}
    \end{minipage} \hfill
    \begin{minipage}{0.3\linewidth}
        \centering
        \includegraphics[width=\linewidth]{images/death_scorpion.png}
        \caption*{Do bọ cạp}
    \end{minipage}
    
    \vspace{0.5cm}
    
    \begin{minipage}{0.4\linewidth}
        \centering
        \includegraphics[width=0.7\linewidth]{images/death_trap.png}
        \caption*{Do rơi vào bẫy}
    \end{minipage}
    
    \caption{Minh họa các trạng thái kết thúc trò chơi.}
    \label{fig:death_states}
\end{figure}

Sau khi chết, màn hình game over xuất hiện từ dưới lên.

\begin{figure}[H]
    \centering
    
    \includegraphics[width=0.5\linewidth]{images/ui_death.png} 
    \caption{Giao diện thông báo khi người chơi chết.}
    \label{fig:game_over}
\end{figure}

Giao diện này cung cấp 4 chức năng điều hướng chính:
\begin{itemize}
    \item \textbf{TRY AGAIN:} Tải lại màn chơi hiện tại.
    \item \textbf{UNDO MOVE:} Hoàn tác bước đi sai lầm cuối cùng để người chơi có thể thử lại hướng di chuyển khác.
    \item \textbf{ABANDON HOPE:} Chấp nhận kết thúc phiên chơi hiện tại và coi hướng dẫn giải.
    \item \textbf{BACK TO MAIN MENU:} Quay trở về giao diện Menu chính của ứng dụng.
\end{itemize}

